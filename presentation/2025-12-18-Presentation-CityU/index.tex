% Options for packages loaded elsewhere
% Options for packages loaded elsewhere
\PassOptionsToPackage{unicode}{hyperref}
\PassOptionsToPackage{hyphens}{url}
\PassOptionsToPackage{dvipsnames,svgnames,x11names}{xcolor}
%
\documentclass[
  letterpaper,
  DIV=11,
  numbers=noendperiod]{scrartcl}
\usepackage{xcolor}
\usepackage{amsmath,amssymb}
\setcounter{secnumdepth}{-\maxdimen} % remove section numbering
\usepackage{iftex}
\ifPDFTeX
  \usepackage[T1]{fontenc}
  \usepackage[utf8]{inputenc}
  \usepackage{textcomp} % provide euro and other symbols
\else % if luatex or xetex
  \usepackage{unicode-math} % this also loads fontspec
  \defaultfontfeatures{Scale=MatchLowercase}
  \defaultfontfeatures[\rmfamily]{Ligatures=TeX,Scale=1}
\fi
\usepackage{lmodern}
\ifPDFTeX\else
  % xetex/luatex font selection
\fi
% Use upquote if available, for straight quotes in verbatim environments
\IfFileExists{upquote.sty}{\usepackage{upquote}}{}
\IfFileExists{microtype.sty}{% use microtype if available
  \usepackage[]{microtype}
  \UseMicrotypeSet[protrusion]{basicmath} % disable protrusion for tt fonts
}{}
\makeatletter
\@ifundefined{KOMAClassName}{% if non-KOMA class
  \IfFileExists{parskip.sty}{%
    \usepackage{parskip}
  }{% else
    \setlength{\parindent}{0pt}
    \setlength{\parskip}{6pt plus 2pt minus 1pt}}
}{% if KOMA class
  \KOMAoptions{parskip=half}}
\makeatother
% Make \paragraph and \subparagraph free-standing
\makeatletter
\ifx\paragraph\undefined\else
  \let\oldparagraph\paragraph
  \renewcommand{\paragraph}{
    \@ifstar
      \xxxParagraphStar
      \xxxParagraphNoStar
  }
  \newcommand{\xxxParagraphStar}[1]{\oldparagraph*{#1}\mbox{}}
  \newcommand{\xxxParagraphNoStar}[1]{\oldparagraph{#1}\mbox{}}
\fi
\ifx\subparagraph\undefined\else
  \let\oldsubparagraph\subparagraph
  \renewcommand{\subparagraph}{
    \@ifstar
      \xxxSubParagraphStar
      \xxxSubParagraphNoStar
  }
  \newcommand{\xxxSubParagraphStar}[1]{\oldsubparagraph*{#1}\mbox{}}
  \newcommand{\xxxSubParagraphNoStar}[1]{\oldsubparagraph{#1}\mbox{}}
\fi
\makeatother


\usepackage{longtable,booktabs,array}
\usepackage{calc} % for calculating minipage widths
% Correct order of tables after \paragraph or \subparagraph
\usepackage{etoolbox}
\makeatletter
\patchcmd\longtable{\par}{\if@noskipsec\mbox{}\fi\par}{}{}
\makeatother
% Allow footnotes in longtable head/foot
\IfFileExists{footnotehyper.sty}{\usepackage{footnotehyper}}{\usepackage{footnote}}
\makesavenoteenv{longtable}
\usepackage{graphicx}
\makeatletter
\newsavebox\pandoc@box
\newcommand*\pandocbounded[1]{% scales image to fit in text height/width
  \sbox\pandoc@box{#1}%
  \Gscale@div\@tempa{\textheight}{\dimexpr\ht\pandoc@box+\dp\pandoc@box\relax}%
  \Gscale@div\@tempb{\linewidth}{\wd\pandoc@box}%
  \ifdim\@tempb\p@<\@tempa\p@\let\@tempa\@tempb\fi% select the smaller of both
  \ifdim\@tempa\p@<\p@\scalebox{\@tempa}{\usebox\pandoc@box}%
  \else\usebox{\pandoc@box}%
  \fi%
}
% Set default figure placement to htbp
\def\fps@figure{htbp}
\makeatother

\ifLuaTeX
  \usepackage{luacolor}
  \usepackage[soul]{lua-ul}
\else
  \usepackage{soul}
\fi

% definitions for citeproc citations
\NewDocumentCommand\citeproctext{}{}
\NewDocumentCommand\citeproc{mm}{%
  \begingroup\def\citeproctext{#2}\cite{#1}\endgroup}
\makeatletter
 % allow citations to break across lines
 \let\@cite@ofmt\@firstofone
 % avoid brackets around text for \cite:
 \def\@biblabel#1{}
 \def\@cite#1#2{{#1\if@tempswa , #2\fi}}
\makeatother
\newlength{\cslhangindent}
\setlength{\cslhangindent}{1.5em}
\newlength{\csllabelwidth}
\setlength{\csllabelwidth}{3em}
\newenvironment{CSLReferences}[2] % #1 hanging-indent, #2 entry-spacing
 {\begin{list}{}{%
  \setlength{\itemindent}{0pt}
  \setlength{\leftmargin}{0pt}
  \setlength{\parsep}{0pt}
  % turn on hanging indent if param 1 is 1
  \ifodd #1
   \setlength{\leftmargin}{\cslhangindent}
   \setlength{\itemindent}{-1\cslhangindent}
  \fi
  % set entry spacing
  \setlength{\itemsep}{#2\baselineskip}}}
 {\end{list}}
\usepackage{calc}
\newcommand{\CSLBlock}[1]{\hfill\break\parbox[t]{\linewidth}{\strut\ignorespaces#1\strut}}
\newcommand{\CSLLeftMargin}[1]{\parbox[t]{\csllabelwidth}{\strut#1\strut}}
\newcommand{\CSLRightInline}[1]{\parbox[t]{\linewidth - \csllabelwidth}{\strut#1\strut}}
\newcommand{\CSLIndent}[1]{\hspace{\cslhangindent}#1}



\setlength{\emergencystretch}{3em} % prevent overfull lines

\providecommand{\tightlist}{%
  \setlength{\itemsep}{0pt}\setlength{\parskip}{0pt}}



 


\KOMAoption{captions}{tableheading}
\makeatletter
\@ifpackageloaded{caption}{}{\usepackage{caption}}
\AtBeginDocument{%
\ifdefined\contentsname
  \renewcommand*\contentsname{Table of contents}
\else
  \newcommand\contentsname{Table of contents}
\fi
\ifdefined\listfigurename
  \renewcommand*\listfigurename{List of Figures}
\else
  \newcommand\listfigurename{List of Figures}
\fi
\ifdefined\listtablename
  \renewcommand*\listtablename{List of Tables}
\else
  \newcommand\listtablename{List of Tables}
\fi
\ifdefined\figurename
  \renewcommand*\figurename{Figure}
\else
  \newcommand\figurename{Figure}
\fi
\ifdefined\tablename
  \renewcommand*\tablename{Table}
\else
  \newcommand\tablename{Table}
\fi
}
\@ifpackageloaded{float}{}{\usepackage{float}}
\floatstyle{ruled}
\@ifundefined{c@chapter}{\newfloat{codelisting}{h}{lop}}{\newfloat{codelisting}{h}{lop}[chapter]}
\floatname{codelisting}{Listing}
\newcommand*\listoflistings{\listof{codelisting}{List of Listings}}
\makeatother
\makeatletter
\makeatother
\makeatletter
\@ifpackageloaded{caption}{}{\usepackage{caption}}
\@ifpackageloaded{subcaption}{}{\usepackage{subcaption}}
\makeatother
\usepackage{bookmark}
\IfFileExists{xurl.sty}{\usepackage{xurl}}{} % add URL line breaks if available
\urlstyle{same}
\hypersetup{
  pdftitle={Network and AI-Driven Approaches in Psychometrics and Behavioral Science: Concepts and Empirical Case Studies},
  pdfauthor={Jihong Zhang*, Ph.D.},
  colorlinks=true,
  linkcolor={blue},
  filecolor={Maroon},
  citecolor={Blue},
  urlcolor={Blue},
  pdfcreator={LaTeX via pandoc}}


\title{Network and AI-Driven Approaches in Psychometrics and Behavioral
Science: Concepts and Empirical Case Studies}
\usepackage{etoolbox}
\makeatletter
\providecommand{\subtitle}[1]{% add subtitle to \maketitle
  \apptocmd{\@title}{\par {\large #1 \par}}{}{}
}
\makeatother
\subtitle{Department of Social and Behavioural Science, CityU, HK}
\author{Jihong Zhang*, Ph.D.}
\date{2025-12-18}
\begin{document}
\maketitle


\subsection{Abstract}\label{abstract}

This research seminar examines network-based approaches in psychometrics
and behavioral science, integrating conceptual foundations with
empirical demonstrations. We first contrast psychological network models
with traditional latent variable frameworks to highlight how network
analysis captures the dynamic interplay among symptoms, behaviors, and
cognitive processes. Two case studies illustrate these advantages: (1)
estimating cognitive networks from EEG data using Gaussian graphical
modeling to characterize functional dependence among psychological
processes; and (2) modeling longitudinal symptom dynamics in eating
disorders with graphical vector autoregression to identify temporal risk
pathways and gender-specific mechanisms of symptom maintenance. These
applications underscore how network methods enhance theoretical
interpretation and inform targeted intervention strategies. The seminar
concludes by discussing emerging innovations in AI-augmented
psychometrics, including large language models and computational tools
that expand the possibilities for measurement, simulation, and adaptive
decision-making in applied settings.

\subsection{About me}\label{about-me}

\begin{figure}

\begin{minipage}{0.25\linewidth}
\pandocbounded{\includegraphics[keepaspectratio]{../../presentation/2025-12-18-Presentation-CityU/figures/Profile_Official.png}}\end{minipage}%
%
\begin{minipage}{0.55\linewidth}
\pandocbounded{\includegraphics[keepaspectratio]{../../presentation/2025-12-18-Presentation-CityU/figures/Career.png}}\end{minipage}%
%
\begin{minipage}{0.20\linewidth}
~\end{minipage}%

\end{figure}%

I'll briefly introduce myself and my current role. I'll touch on recent
projects and how they connect to today's talk.

\subsection{Research Path}\label{research-path}

\pandocbounded{\includegraphics[keepaspectratio]{../../presentation/2025-12-18-Presentation-CityU/figures/research_wordcloud2025.png}}

My research mainly focus on quantitative methods in psychology, such as
psychological network analysis, diagnostic measurement, structural
equation modeling, and AI measurement. Besides psychometric methods, my
empirical research domains cover cognitive process of students,
psychopathology topics like emotion or eating disorders, academic
motivation, or AI assessment

\subsection{Presentation Outline}\label{presentation-outline}

\subsubsection{30 minutes}\label{minutes}

\begin{itemize}
\item
  Part \textbf{I}: Foundations of network analysis
\item
  Part \textbf{II}: Psychological networks in theory and practice
\item
  Part \textbf{III}: Empirical applications: EEG-based networks and
  longitudinal eating disorder networks
\item
  Part \textbf{IV}: Emerging directions: AI-augmented psychometric
  research
\end{itemize}

\subsubsection{15 minutes}\label{minutes-1}

\begin{itemize}
\tightlist
\item
  Part \textbf{V}: Q\&A
\end{itemize}

I'll walk through four parts: a quick primer on network analysis, why
psychological networks are useful, a case study on eating disorders
using longitudinal networks, and then questions. The goal is to build
the concept of network models first, then show practical results.

\begin{enumerate}
\def\labelenumi{\arabic{enumi}.}
\item
  Illustrate the conceptual foundations of network
\item
  Demonstrate two empirical case studies:

  \begin{itemize}
  \tightlist
  \item
    EEG-based networks to examine brain activities with self-reported
    mediation quality
  \item
    Longitudinal eating disorder networks with graphical vector
    autoregression, including gender-specific risk pathways
  \end{itemize}
\item
  Conclude with emerging innovations in AI-augmented psychometric
  research
\end{enumerate}

\section{What Is Network Analysis}\label{what-is-network-analysis}

\subsection{Network Analysis in
General}\label{network-analysis-in-general}

Network analysis is a broad area. It has many names in varied fields:

\begin{enumerate}
\def\labelenumi{\arabic{enumi}.}
\item
  Graphical Model (Computer Science, Machine Learning)

  \begin{itemize}
  \tightlist
  \item
    History: Statistical physics, such as large system of particles
    (Lauritzen, 1996)
  \end{itemize}
\end{enumerate}

\begin{enumerate}
\def\labelenumi{\arabic{enumi}.}
\setcounter{enumi}{1}
\tightlist
\item
  Bayesian Network (Computer Science, Educational Measurement)
\end{enumerate}

\pandocbounded{\includegraphics[keepaspectratio]{figures/bayesian_network_example.webp}}

\begin{enumerate}
\def\labelenumi{\arabic{enumi}.}
\setcounter{enumi}{2}
\tightlist
\item
  Social Network (Sociology, Social psychology)
\end{enumerate}

\begin{figure}

\centering{

\pandocbounded{\includegraphics[keepaspectratio]{../../teaching/2024-01-12-syllabus-adv-multivariate-esrm-6553/Images/Lecture12_Network/SocialNetwork.png}}

}

\caption{\label{fig-social-network}The Taste for Privacy: An Analysis of
College Student Privacy Settings in an Online Social Network (Lewis et
al., 2008). Facebook friendship network in a single undergraduate dorm.
Red nodes represent students with private profiles; yellow nodes
represent students with public profiles.}

\end{figure}%

\begin{center}\rule{0.5\linewidth}{0.5pt}\end{center}

\begin{enumerate}
\def\labelenumi{\arabic{enumi}.}
\setcounter{enumi}{3}
\tightlist
\item
  \textbf{Latent Factor Model or Structural Equation Model} (Psychology,
  Education)
\end{enumerate}

\begin{enumerate}
\def\labelenumi{\arabic{enumi}.}
\setcounter{enumi}{4}
\tightlist
\item
  \textbf{Psychological Network Analysis} (Psychopathology, Psychology)
\end{enumerate}

\begin{figure}

\centering{

\pandocbounded{\includegraphics[keepaspectratio]{../../teaching/2024-01-12-syllabus-adv-multivariate-esrm-6553/Images/Lecture12_Network/GGM.png}}

}

\caption{\label{fig-GGM}Exploratory factor analysis and psychological
network analysis of Big Five personality (Borsboom et al., 2021)}

\end{figure}%

Note that many fields use network-like models but with different names
and goals. The unifying idea is variables as nodes and relationships as
edges. Psychological networks are one specific approach within this
broader space.

\subsection{Commonality: Key Components in the
network}\label{commonality-key-components-in-the-network}

All network models share common structural elements:

\begin{enumerate}
\def\labelenumi{\arabic{enumi}.}
\item
  \textbf{Nodes} (vertices): Random variables, individuals, or units of
  analysis
\item
  \textbf{Edges} (links): Relationships or connections between nodes

  \begin{itemize}
  \tightlist
  \item
    \textbf{Directed} vs.~\textbf{Undirected}
  \item
    \textbf{Weighted} vs.~\textbf{Unweighted}
  \end{itemize}
\item
  \textbf{Adjacency matrix}: Mathematical representation of the network

  \begin{itemize}
  \tightlist
  \item
    Rows and columns represent nodes
  \item
    Entries indicate presence/strength of edges
  \end{itemize}
\item
  \textbf{Paths} and \textbf{Cycles}: Sequences of connected nodes
\end{enumerate}

Before diving into how networks differ, let me clarify the universal
building blocks that all network models share:

\begin{enumerate}
\def\labelenumi{\arabic{enumi}.}
\item
  \textbf{Nodes} (also called vertices) represent the units of analysis.
  Depending on the network type, nodes can be variables (psychological
  symptoms, cognitive abilities), individuals (people in social
  networks), or other entities. The choice of what constitutes a node is
  theory-driven and depends on your research question.
\item
  \textbf{Edges} (also called links or connections) represent
  relationships between nodes. Edges can be:

  \begin{itemize}
  \tightlist
  \item
    \textbf{Directed} (arrows, indicating asymmetric relationships like
    A → B) or \textbf{Undirected} (lines, indicating symmetric
    relationships like A --- B)
  \item
    \textbf{Weighted} (with numerical values indicating strength of
    relationship) or \textbf{Unweighted} (binary presence/absence of
    connection)
  \end{itemize}
\item
  \textbf{Adjacency matrix} is the mathematical representation of the
  network. It's a square matrix where rows and columns represent nodes,
  and entries indicate whether edges exist (and their weights if
  applicable). For example, entry (i,j) represents the edge from node i
  to node j. This matrix is symmetric for undirected networks.
\item
  \textbf{Paths} are sequences of nodes connected by edges (e.g., A → B
  → C). \textbf{Cycles} are paths that return to the starting node
  (e.g., A → B → C → A). These concepts are important for understanding
  how effects can propagate through the network and create feedback
  loops.
\end{enumerate}

These universal components provide the language for describing network
structure, regardless of what the nodes and edges represent in your
specific application.

\subsection{Uniqueness: Statistical dependence \&
Purpose}\label{uniqueness-statistical-dependence-purpose}

Different network models represent distinct types of statistical
dependence:

\begin{enumerate}
\def\labelenumi{\arabic{enumi}.}
\item
  \textbf{Bayesian networks} (DAG): \textbf{Conditional dependence} P(X
  \textbar{} Parents)
\item
  \textbf{Social networks}: \textbf{Observed social relationships} among
  social units
\item
  \textbf{Latent factor models}: \textbf{Marginal dependence} via common
  latent causes

  \begin{itemize}
  \tightlist
  \item
    Observed variables are \textbf{conditionally independent} given
    latent factors
  \end{itemize}
\item
  \textbf{Psychological networks}: \textbf{Conditional dependence given
  all other variables}

  \begin{itemize}
  \tightlist
  \item
    Partial correlations (cross-sectional) or lagged effects (temporal)
  \end{itemize}
\end{enumerate}

This slide highlights the key statistical distinction across network
frameworks:

\begin{enumerate}
\def\labelenumi{\arabic{enumi}.}
\item
  Bayesian networks encode conditional dependence through directed
  edges. Each variable depends on its parents via conditional
  probability distributions P(X\textbar Parents). Edges represent direct
  causal influences in the assumed structure model, making them useful
  for causal inference.
\item
  Social networks model observed relationships among social units
  (people, organizations) rather than statistical dependence between
  random variables. Edges represent connections like friendships or
  collaborations.
\item
  Latent factor models explain marginal dependence (correlations)
  through common latent causes. The crucial property is that observed
  variables are conditionally independent given the latent factors. We
  called that local independence. This is the fundamental assumption
  that distinguishes latent factor models.
\item
  Psychological networks capture conditional dependence given all other
  variables in the system. Edges represent partial correlations
  (controlling for all other nodes) in Gaussian graphical models, or
  lagged regression coefficients in temporal models (GVAR). No latent
  variables are assumed---the network itself is the explanatory model.
\end{enumerate}

The key takeaway: These different dependence structures fundamentally
change how we interpret edges and what questions each approach can
answer. Choosing the right framework depends on your theoretical
assumptions about what generates the observed data.

\subsection{Psychological Networks and Network
Psychometrics}\label{psychological-networks-and-network-psychometrics}

\pandocbounded{\includegraphics[keepaspectratio]{figures/network_book_cover.png}}

\begin{enumerate}
\def\labelenumi{\arabic{enumi}.}
\item
  \textbf{Network psychometrics} is a novel psychometric area that
  represents complex phenomena of measured constructs as sets of
  elements that interact with each other (Isvoranu et al., 2022).
\item
  It is inspired by the {mutualism model} and research in ecosystem
  modeling (Kan et al., 2019).

  \begin{itemize}
  \tightlist
  \item
    The mutualism model proposes that basic cognitive abilities directly
    and positively interact during development.
  \end{itemize}
\item
  In recent years, interest in psychological networks as \ul{dynamics or
  reciprocal causation} among variables receive more attention.
\end{enumerate}

Network psychometrics treats symptoms or behaviors as mutually
interacting parts of a system. Inspired by mutualism, we look for
reinforcing cycles. This viewpoint is useful when feedback loops, not
latent traits alone, drive what we observe.

\section{Why use psychological
networks?}\label{why-use-psychological-networks}

\subsection{Latent Factor Model VS. Psychological Network
Model}\label{latent-factor-model-vs.-psychological-network-model}

\begin{itemize}
\item
  \ul{\textbf{Latent factor model}} (common factor model) assumes that
  associations between observed features can be explained by one or more
  common factors (e.g., agreebleness, conscientiousness).
\item
  \ul{\textbf{Psychometric networks}}, however, assume that associations
  between observed features are the reason for the development of one
  system. In this view, ``personality'' is the network itself.
\end{itemize}

\begin{figure}[H]

{\centering \pandocbounded{\includegraphics[keepaspectratio]{../../teaching/2024-01-12-syllabus-adv-multivariate-esrm-6553/Images/Lecture12_Network/GGM.png}}

}

\caption{Exploratory factor analysis and psychological network analysis
of Big Five personality (Borsboom et al., 2021)}

\end{figure}%

Contrast: factor models explain covariation via common causes; network
models explain it via direct interactions among observed features. I'll
emphasize when each is useful and how the interpretation of
``personality'' differs between the two views.

\subsection{Common factor or
Mutualism?}\label{common-factor-or-mutualism}

\begin{itemize}
\item
  ``\textbf{Openness}'' dimension:

  \begin{itemize}
  \tightlist
  \item
    Q5 (Creativity). \ldots{} comes up with new ideas: Disagree (1) to
    Agree (5) \footnote{Source:
      https://arc.psych.wisc.edu/self-report/big-five-inventory-bfi/}
  \item
    Q10 (Curiosity). \ldots{} curious about lots of different things:
    Disagree (1) to Agree (5)
  \end{itemize}
\item
  Does ``openness'' really exist?
\end{itemize}

\begin{figure}

\begin{minipage}{0.28\linewidth}

\begin{figure}[H]

\centering{

\pandocbounded{\includegraphics[keepaspectratio]{../../presentation/2025-12-18-Presentation-CityU/figures/common_factor.png}}

}

\caption{\label{fig-commonality}Common factor}

\end{figure}%

\end{minipage}%
%
\begin{minipage}{0.28\linewidth}

\begin{figure}[H]

\centering{

\pandocbounded{\includegraphics[keepaspectratio]{../../presentation/2025-12-18-Presentation-CityU/figures/multualism.png}}

}

\caption{\label{fig-mutualism}Mutualism}

\end{figure}%

\end{minipage}%
%
\begin{minipage}{0.44\linewidth}
~\end{minipage}%

\end{figure}%

I'll use ``Openness'' as an example. Is the shared variance due to a
latent factor, or do the items reinforce each other over time? The
mutualism perspective suggests growth through interactions among
observed skills or traits.

\subsection{Research aims: Density, Centrality, Pathways, and Group
Differences}\label{research-aims-density-centrality-pathways-and-group-differences}

\textbf{Aim 1. Network Density}

Assess overall connectivity and sparsity (system activation likelihood)

\pandocbounded{\includegraphics[keepaspectratio]{figures/EEG/Network_Sparsity.png}}

\textbf{Aim 2. Identify Pathways}

Map symptom interactions and feedback loops (A ↔ B ↔ C ↔ A)
\pandocbounded{\includegraphics[keepaspectratio]{figures/EEG/pathway.png}}

\begin{center}\rule{0.5\linewidth}{0.5pt}\end{center}

\textbf{Aim 3. Intervention Targets}

Identify the most central/influential nodes for treatment

\pandocbounded{\includegraphics[keepaspectratio]{figures/EEG/Network_Centrality.png}}

\textbf{Aim 4. Group Differences}

Compare network structures across populations

\pandocbounded{\includegraphics[keepaspectratio]{figures/EEG/Network_Comparison.png}}

These analytical goals help bridge theory and practice, guiding both our
understanding of psychological phenomena and clinical decision-making.

Psychological network analysis serves multiple applied and theoretical
goals:

\begin{enumerate}
\def\labelenumi{\arabic{enumi}.}
\item
  \textbf{Identify pathways}: We can map how symptoms or behaviors
  influence each other over time, including feedback loops (e.g.,
  insomnia → fatigue → worry → insomnia). Understanding these pathways
  reveals mechanisms of symptom maintenance.
\item
  \textbf{Group differences}: We can compare network structures across
  different populations (e.g., by gender, diagnosis, treatment response)
  to understand how symptom dynamics differ and tailor interventions
  accordingly.
\item
  \textbf{Intervention targets}: By calculating node centrality
  measures, we identify which symptoms are most influential in the
  network. Targeting these central nodes in treatment may have cascading
  positive effects on the entire system.
\item
  \textbf{Network density}: A denser network (more edges) indicates
  stronger interconnections among symptoms. Individuals with denser
  networks may be more vulnerable because activating one symptom more
  easily triggers others, increasing risk for comorbidity.
\item
  \textbf{Symptom communities}: Clustering algorithms can identify
  groups of symptoms that tend to co-occur, revealing distinct
  sub-syndromes within broader diagnostic categories. This can inform
  dimensional approaches to psychopathology.
\end{enumerate}

\subsection{Data types: Cross-sectional vs.~Longitudinal
network}\label{data-types-cross-sectional-vs.-longitudinal-network}

\begin{enumerate}
\def\labelenumi{\arabic{enumi}.}
\item
  \textbf{Cross-sectional network}

  \begin{itemize}
  \tightlist
  \item
    Data: Multivariate cross-section data
  \item
    Model: Gaussian graphical model (GGM)
  \item
    Dependence: The variables depend on other variables at the same time
  \item
    Edge Statistics: Partial Correlation (\(r\))
  \item
    Methodology: Correlation analysis
  \end{itemize}
\end{enumerate}

\pandocbounded{\includegraphics[keepaspectratio]{figures/Introduction/undirect_formula.png}}

\begin{enumerate}
\def\labelenumi{\arabic{enumi}.}
\setcounter{enumi}{1}
\item
  \textbf{Longitudinal network (today's focus)}

  \begin{itemize}
  \tightlist
  \item
    Data: Multivariate time-seires data
  \item
    Model: Graphical vector autoregressive (VAR) model
  \item
    Dependence: The values of the variables at time \(t\) depend
    linearly on the values at time \(t -1\)
  \item
    Edge Statistics: Lag-1 regression coefficient (\(\beta\))
  \item
    Methodology: Granger-causality
  \end{itemize}
\end{enumerate}

\pandocbounded{\includegraphics[keepaspectratio]{figures/Introduction/Longitudinal_Network_Formula.png}}

\section{Case Study: A longitudinal Electroencephalography (EEG) network
analysis}\label{case-study-a-longitudinal-electroencephalography-eeg-network-analysis}

\pandocbounded{\includegraphics[keepaspectratio]{figures/eeg-private.jpg}}

\subsection{Brain Activity Networks}\label{brain-activity-networks}

The human brain is one of the most complex networks in the world.

-- Farahani et al. (2019)

\pandocbounded{\includegraphics[keepaspectratio]{figures/EEG/brain_network_procedure.png}}

studies on its static and dynamic properties have undergone explosive
growth in recent years

\subsection{Research background}\label{research-background}

\begin{itemize}
\item
  The team utilize a meditation training called {Focused Attention (FA)
  meditation} to improve participants' meditation quality.
\item
  \textbf{Key question}:

  \begin{itemize}
  \tightlist
  \item
    {What is the relationship between brain activities and participants'
    meditation outcomes.}
  \end{itemize}
\item
  Why use network analysis for EEG data?
\end{itemize}

\begin{enumerate}
\def\labelenumi{\arabic{enumi}.}
\tightlist
\item
  \textbf{Functional connectivity}: Brain regions interact and
  coordinate during cognitive processes
\item
  \textbf{Brain dynamics}: Alpha and theta rhythms reflect neural
  communication across brain regions
\end{enumerate}

\begin{enumerate}
\def\labelenumi{\arabic{enumi}.}
\setcounter{enumi}{2}
\tightlist
\item
  \textbf{Topographical patterns}: Multiple scalp regions (frontal or
  posterior) form interconnected systems
\item
  \textbf{Temporal changes}: Longitudinal tracking reveals how brain
  networks evolve with meditation training
\end{enumerate}

Network analysis is well-suited for EEG data in meditation research for
several reasons:

\begin{enumerate}
\def\labelenumi{\arabic{enumi}.}
\item
  \textbf{Functional connectivity}: The brain operates as a complex
  network where different regions interact and coordinate during
  cognitive processes like meditation. Network models can capture these
  dependencies among brain regions, revealing how they work together to
  produce meditative states.
\item
  \textbf{Brain dynamics}: EEG measures brain activity (alpha, theta
  waves). These rhythms don't occur in isolation---they reflect neural
  communication across brain regions. Network analysis can model these
  patterns and identify which regions are most strongly coupled with
  meditation process.
\item
  \textbf{Topographical patterns}: We're measuring activity across
  multiple scalp regions simultaneously (frontal, temporal-central,
  posterior). Rather than analyzing each region independently, network
  analysis captures the system-level connectivity, such as which regions
  is the functional hubs, how information flows between regions, and
  which connections are strongest.
\item
  \textbf{Temporal changes}: With 24 sessions over 8 weeks, we can use
  longitudinal network methods (like GVAR) to examine how brain network
  structure evolves with meditation training.
\end{enumerate}

By treating EEG signals from different brain regions as nodes in a
network, we can identify the functional architecture of meditation and
how it develops with practice.

\subsection{Data}\label{data}

\begin{itemize}
\item
  \textbf{Sample}: 19 participants completed 8 weeks of FA training with
  3 visits per week, resulting in 24 sessions.
\item
  \textbf{Training}: in each session, they completed 20-minutes of
  standardized audio-guided Focus Attention meditation practice. EEG
  data was collected
\item
  \textbf{Measure}: Self-reported State Mindfulness Scale (SMS\_Mind)
  and body (SMS\_Body)
\end{itemize}

\begin{figure}[H]

{\centering \includegraphics[width=1\linewidth,height=\textheight,keepaspectratio]{figures/Test2_SMSBody_Trend.jpg}

}

\caption{Self-reported State Mindfulness Scale Scores over 24 sessions}

\end{figure}%

Alpha Power (8-12 Hz)

State: Relaxed wakefulness, quiet contemplation, drowsiness, eyes
closed.

Function: Associated with suppressing irrelevant sensory input, internal
focus, proactive cognitive control, and memory encoding/retrieval.

Changes: Decreases (desynchronizes) during focused attention or
demanding tasks, increases when relaxed.

Theta Power (4-8 Hz)

State: Drowsiness, light sleep, REM sleep, deep relaxation, creativity.

Function: Linked to memory formation, navigation, mental effort,
executive functions, and threat prediction.Changes: Increases with
cognitive effort, drowsiness, and during sleep; inversely related to
alpha in some tasks

\subsection{Nodes}\label{nodes}

\begin{itemize}
\item
  {2 State Mindfulness Nodes}

  \begin{itemize}
  \item
    \textbf{SMS\_Mind}: the subjective quality of mindfulness of mind
    (e.g., thoughts, emotions, mental acuity)
  \item
    \textbf{SMS\_Body}: the subjective quality of mindfulness of body
    (e.g., physical and bodily sensations)
  \end{itemize}
\item
  {6 EEG Nodes}: \textbf{Alpha} (8-12 Hz; relaxed wakefulness) and
  \textbf{Theta} (4-8 Hz; light sleep) power separated across respective
  frontal, temporal-central, and posterior regions

  \begin{itemize}
  \tightlist
  \item
    {frontal} alpha
  \item
    {frontal} theta
  \item
    {temporal-central} alpha
  \item
    {temporal-central} theta
  \item
    {posterior} alpha
  \item
    {posterior} theta
  \end{itemize}
\end{itemize}

\subsection{Results: regularization}\label{results-regularization}

{Unregularized} psychological network

\pandocbounded{\includegraphics[keepaspectratio]{figures/EEG/Present_FA_temporal_network_str_full.jpg}}

{Regularized} psychological network

\pandocbounded{\includegraphics[keepaspectratio]{figures/EEG/Present_FA_temporal_network_str.jpg}}

{Alpha power}; {Theta power}; {Self-report mindfulness}

\subsection{Discussion}\label{discussion}

\pandocbounded{\includegraphics[keepaspectratio]{figures/EEG/Present_FA_temporal_network_str.jpg}}

\begin{itemize}
\item
  \textbf{Brain is dynamical}

  \begin{enumerate}
  \def\labelenumi{\arabic{enumi}.}
  \tightlist
  \item
    e.g., Posterior alpha can enhance frontal theta over time
  \end{enumerate}
\item
  \textbf{Brain region (topography) matters}

  \begin{enumerate}
  \def\labelenumi{\arabic{enumi}.}
  \item
    e.g., Frontal alpha power enhances mindfulness, while posterior
    alpha power suppresses mindfulness of body and mind over time.
  \item
    e.g., Posterior theta power enhances mindfulness, while
    temporal-central theta power suppresses mindfulness
  \end{enumerate}
\item
  \textbf{Identify ideal brain states}

  \begin{enumerate}
  \def\labelenumi{\arabic{enumi}.}
  \tightlist
  \item
    e.g., According to centrality measures, frontal alpha and posterior
    theta are most important EEG signal for FA meditation training.
  \end{enumerate}
\end{itemize}

If you have high frontal alpha and posterior theta, you tend to have
better meditation experience.

\subsection{Conclusion}\label{conclusion}

\begin{itemize}
\item
  In neuroscience, psychological network could be very useful to
  understand brain activity patterns and their relationships with
  external behaviours.
\item
  But, can we use it to understand the complex mental disorders? Symptom
  network seems a very interesting idea.
\end{itemize}

\section{Case Study: Gender Differences for Eating Disorder Symptom
Networks}\label{case-study-gender-differences-for-eating-disorder-symptom-networks}

\subsection{Research background}\label{research-background-1}

\begin{itemize}
\item
  The present study examined sex-specific, symptom-level relationships
  among emotion regulation (ER), interpersonal problems (IP), and eating
  disorder (ED) psychopathology in a large sample of Chinese adolescents
  (Zhang et al., 2024).
\item
  \textbf{Background}: Eating disorders are serious issues for college
  students. The eating disorders symptoms are dangerous and co-occur
  with other psychological issues.
\item
  \textbf{Motivation}: Given the complex relationships between eating
  disorders with other risky behaviors, we need a novel model to
  untangle those interplay which can help with further intervention.
\item
  Assumption: {Eating disorders combined with other risky factors be
  considered a network?}
\end{itemize}

\pandocbounded{\includegraphics[keepaspectratio]{figures/eating_disorders.jpg}}

I'll first talk about the background and research questions. I will talk
about the rationales of each RQ later.

\subsection{Eating Disorder Network?}\label{eating-disorder-network}

\includegraphics[width=1\linewidth,height=\textheight,keepaspectratio]{figures/structure.jpg}

\begin{enumerate}
\def\labelenumi{\arabic{enumi}.}
\item
  Interrelationships among components---emotion regulation,
  interpersonal problems, and eating disorders:

  \begin{itemize}
  \item
    \textbf{Emotion regulation theory} suggests that difficulties in
    emotion regulation can result in ED behaviors.
  \item
    \textbf{Interpersonal psychotherapy theory} posits that
    interpersonal problems may exacerbate ED (Murphy et al., 2012).
  \item
    Empirical studies consider these three to constitute an
    ``ecosystem'' (Ambwani et al., 2014). Emotion regulation and
    interpersonal functioning exhibit reciprocal effects on the
    maintenance of ED.
  \end{itemize}
\end{enumerate}

\begin{itemize}
\item
  {Advantages of longitudinal network}

  \begin{enumerate}
  \def\labelenumi{\arabic{enumi}.}
  \tightlist
  \item
    It considers symptoms cause each other over time.
  \item
    It allows us to identify some symptoms may play the most important
    roles across time points
  \item
    It allows sex-specific developmental patterns of eating disorders
  \end{enumerate}
\end{itemize}

\textbf{The motivation is to obtain a holistic picture of the eating
disorders ecosystem.} \textbf{However, the symptom-level dynamics of
eating disorders have not been well investigated.}

I'll motivate ED as an ecosystem involving emotion regulation and
interpersonal problems. Theory and evidence suggest reciprocal effects.
We want to see symptom-level dynamics rather than only broad trait
scores.

Longitudinal psychological networks fit this problem well. They treat ED
behaviors and related traits as interrelated nodes and help us find
impactful symptoms while separating temporal from concurrent effects.

\subsection{Group Comparison and Research
Questions}\label{group-comparison-and-research-questions}

\begin{itemize}
\item
  In network analysis, groups can be compared from three aspects:

  \begin{enumerate}
  \def\labelenumi{\arabic{enumi}.}
  \item
    \textbf{Network structure} (e.g., some nodes connected in group A
    but not in group B).
  \item
    \textbf{Node-level measures}: node centrality (importance) or node
    bridging strength (e.g., some nodes may be more connected to other
    communities).
  \item
    \textbf{Network edge weights} (e.g., node 1 and node 2 may have a
    strong relationship in group A but a weaker relationship in group
    B).
  \end{enumerate}
\item
  Research Questions:

  \begin{enumerate}
  \def\labelenumi{\arabic{enumi}.}
  \tightlist
  \item
    Are there gender differences in the \textbf{network structures} of
    eating disorder longitudinal networks?
  \item
    Are there gender differences in the \textbf{network importance} of
    longitudinal networks?
  \item
    Are there gender differences in \textbf{key paths} of longitudinal
    networks?
  \end{enumerate}
\end{itemize}

Group comparisons can target structure, node-level importance, and
specific edges. I'll preview how we test these differences and what they
imply for tailored interventions.

We ask three questions: do networks differ by gender in overall
features, in which nodes connect, and in which nodes are most
influential or act as bridges?

\subsection{Data \& Measures}\label{data-measures}

\begin{itemize}
\item
  {Sample}

  \begin{itemize}
  \item
    Four waves of data were collected over 18 months.
  \item
    For each wave, demographic information and self-reports on three
    questionnaires (emotion regulation, interpersonal problems, and
    eating disorders) were collected from 1,652 high school students in
    China.
  \item
    After data cleaning, N = 1,540 remained, including 53.9\% girls and
    46.1\% boys.
  \item
    Ages ranged from 11 to 17 years, with a mean of 15.2 years.
  \end{itemize}
\item
  {Measure}

  \begin{enumerate}
  \def\labelenumi{\arabic{enumi}.}
  \tightlist
  \item
    \textbf{Emotion regulation}: Difficulties in Emotion Regulation
    Scale (DERS-18). Six subscales measure different aspects of emotion
    dysregulation: Awareness, Clarity, Goals, Non-acceptance, Impulse,
    Strategies.
  \item
    \textbf{Interpersonal problems}: Inventory of Interpersonal
    Problems---Short Circumplex (IIP-SC). Eight subscales measure varied
    aspects of interpersonal problems (e.g., domineering, cold,
    avoidant).
  \item
    \textbf{Eating disorders}: 12-item short form of the Eating Disorder
    Examination Questionnaire (EDE-QS). Twelve items measure different
    disordered eating behaviors.
  \end{enumerate}
\end{itemize}

We have 26 nodes in the initial networks.

The study follows high school students across four waves over 18 months.
After cleaning, we analyze 1,540 students with a balanced gender split
and typical adolescent ages.

We include emotion regulation subscales, interpersonal problem
subscales, and ED items. This mix lets us see how emotional and social
factors interact with specific ED behaviors.

\subsection{Results}\label{results}

\subsubsection{Temporal Network Structure (left: boys; right:
girls)}\label{temporal-network-structure-left-boys-right-girls}

{Emotion Disorder}; {Interpersonal Problems}; {Eating Regulation}

\begin{figure}[H]

{\centering \includegraphics[width=1\linewidth,height=\textheight,keepaspectratio]{figures/boys_temporal_MG.jpg}

}

\caption{Boys' temporal network}

\end{figure}%

\begin{figure}[H]

{\centering \includegraphics[width=1\linewidth,height=\textheight,keepaspectratio]{figures/girls_temporal_MG.jpg}

}

\caption{Girls' temporal network}

\end{figure}%

These plots show temporal connections. I'll point out shared patterns
and key differences by gender, setting up the summary tables on the next
slides.

\subsection{Overall Network Structure}\label{overall-network-structure}

\subsubsection{Correlation Stability}\label{correlation-stability}

\begin{enumerate}
\def\labelenumi{\arabic{enumi}.}
\item
  The multigroup network stability statistics were acceptable for
  contemporaneous and between-subject networks according to the
  criterion of CS coefficients above 0.7.
\item
  The temporal network was less stable, with CS coefficients ranging
  from 0.51 to 0.56. Less stable indicates higher residual errors
  overall.
\end{enumerate}

\subsubsection{Boys}\label{boys}

\begin{enumerate}
\def\labelenumi{\arabic{enumi}.}
\tightlist
\item
  Sparsity: \textbf{8.06\%} of non-zero edges (temporal)
\item
  Strength: Mean (SD) of edge weights is 0.127 (0.094).
\item
  Node \emph{weight/shape preoccupation} (EDE-WP) exhibited the highest
  \textbf{InStrength}
\item
  Node \emph{weight/shape dissatisfaction} (EDE-WD) exhibited the
  highest \textbf{OutStrength}
\item
  Node \emph{Awareness} (Awr) exhibited the highest \textbf{bridge
  strength}
\end{enumerate}

\subsubsection{Girls}\label{girls}

\begin{enumerate}
\def\labelenumi{\arabic{enumi}.}
\tightlist
\item
  Sparsity: \textbf{10.60\%} of non-zero edges (temporal)
\item
  Strength: Mean (SD) of edge weights is 0.128 (0.102).
\item
  Node \emph{weight/shape preoccupation} (EDE-WP) exhibited the highest
  \textbf{InStrength}
\item
  Node \emph{weight/shape dissatisfaction} (EDE-WD) exhibited the
  highest \textbf{OutStrength}
\item
  Node \emph{weight/shape dissatisfaction} (EDE-WD) exhibited the
  highest \textbf{bridge strength}
\end{enumerate}

Contemporaneous and between-person layers are stable; temporal effects
are moderately stable. I'll note where to be cautious when interpreting
lagged edges.

Boys and girls have similar average temporal strengths, but different
bridging. For boys, Awareness bridges communities; for girls,
Weight/Shape Dissatisfaction plays that role. This suggests different
intervention targets.

\subsection{Network Hub differences between sex
groups}\label{network-hub-differences-between-sex-groups}

\begin{figure}[H]

{\centering \pandocbounded{\includegraphics[keepaspectratio]{figures/Node_Strength_Table.png}}

}

\caption{Centrality measures: Boys vs.~Girls}

\end{figure}%

\begin{enumerate}
\def\labelenumi{\arabic{enumi}.}
\tightlist
\item
  \emph{Long periods without eating} (EDE-WE) and \emph{Food
  preoccupation} (EDE-FP) have significant sex differences in node
  out-strength.
\item
  \emph{Weight/shape control by vomiting or taking laxatives} (EDE-VT)
  has significant sex differences in node in-strength.
\end{enumerate}

The table compares centrality by sex. Bootstrap tests indicate
meaningful differences for specific ED items. I'll highlight how
out-strength and in-strength map to potential symptom drivers and
receivers.

\subsection{Network conncetor difference between sex
groups}\label{network-conncetor-difference-between-sex-groups}

\begin{figure}[H]

{\centering \pandocbounded{\includegraphics[keepaspectratio]{figures/Bridge_Strength_Table.png}}

}

\caption{Bridge strength measures how strongly a node connects to nodes
in different communities or clusters within the network.}

\end{figure}%

\begin{enumerate}
\def\labelenumi{\arabic{enumi}.}
\tightlist
\item
  \emph{Awareness} (Awr) and \emph{Goals} (Gls) have significant sex
  differences.
\item
  \emph{Weight/shape preoccupation} (EDE-WP) and \emph{Binge eating
  episode} (EDE-BE) have significant sex differences.
\end{enumerate}

\textbf{Target nodes for intervention on comorbidity.}

Bridge strength findings identify nodes that connect ED with emotional
and interpersonal domains. Differences suggest gender-specific pathways
to comorbidity and potential targets for reducing cross-domain
activation.

\subsection{Discussion}\label{discussion-1}

\subsubsection{Group Commonalities}\label{group-commonalities}

\begin{enumerate}
\def\labelenumi{\arabic{enumi}.}
\tightlist
\item
  Emotion dysregulation has consistent interconnections with eating
  disorders and interpersonal problems across genders.
\end{enumerate}

\subsubsection{Group Differences}\label{group-differences}

\begin{enumerate}
\def\labelenumi{\arabic{enumi}.}
\tightlist
\item
  Overall network structures of boys and girls are significantly
  different.
\item
  Overall, for \textbf{boys}, the most important bridging nodes were
  \textbf{awareness} and \textbf{nonacceptance} of the DERS, while
  weight/shape preoccupation and domineering/controlling emerged as the
  most central nodes.
\item
  For \textbf{girls}, \textbf{weight/shape dissatisfaction} was
  identified as the most central symptom and the strongest bridging
  node.
\end{enumerate}

Common patterns: similar overall temporal impact, strong coupling among
ED behaviors, and central roles for weight/shape concerns. These are
consistent targets across genders.

Differences: denser girls' networks and different bridge nodes imply
higher comorbidity risk for girls and distinct maintenance pathways for
boys. I'll connect these to tailored clinical strategies.

\subsection{Conclusion}\label{conclusion-1}

\begin{itemize}
\item
  Network psychometrics seems very promising to understand complex
  psychological phenomenon.
\item
  Is there any other novel psychometric areas? My answer is AI-driven
  psychometrics.
\end{itemize}

\section{Further studies: AI
psychometrics}\label{further-studies-ai-psychometrics}

AI-enhanced psychometrics is becoming more and more important.

I shared two of my recent projects with you.

\subsection{AI data augmentation}\label{ai-data-augmentation}

\begin{itemize}
\item
  One of my recent project let AI participants can serve as the data
  points in psychometric assessment given enough accurate human
  information.
\item
  Interview-informed large language models can align with real human
  responses regarding survey responses very well.
\end{itemize}

\begin{figure}[H]

{\centering \pandocbounded{\includegraphics[keepaspectratio]{figures/Future_AI/Figure3_output_BREQ_means.png}}

}

\caption{Behavioral Regulation in Exercise Questionnaire}

\end{figure}%

\begin{center}\rule{0.5\linewidth}{0.5pt}\end{center}

\subsection{AI conversational assessment: One in-class quiz
example}\label{ai-conversational-assessment-one-in-class-quiz-example}

\pandocbounded{\includegraphics[keepaspectratio]{figures/AI_Get_Started.png}}

\pandocbounded{\includegraphics[keepaspectratio]{figures/AI_Feedback.png}}

We need better construct alignment across genders and more simulation
work for inference in multi-group longitudinal networks.

Interpretation challenges remain: what do density differences mean? How
should we read group differences in centrality and edges in applied
contexts? These are active areas for methodological development.

\subsection{Wrap-up}\label{wrap-up}

\begin{enumerate}
\def\labelenumi{\arabic{enumi}.}
\item
  We talk about the network analysis framework
\item
  Two case studies showing network analysis can be widely applied to
  different research areas
\item
  We need novel AI-driven methodology in AI era because AI can serve as
  research tools or data collection tools.
\end{enumerate}

\section{Q\&A}\label{qa}

\subsection{Thank you.}\label{thank-you.}

Let me know if you have any questions.

You can also contact me via \ul{\textbf{jzhang@uark.edu}}

Thank you for your attention. You can reach me by email for follow-ups.
I'm happy to take questions now.

\subsection{Reference}\label{reference}

References are provided for methods, measures, and prior findings
mentioned today. I'm happy to share code and additional materials upon
request.

\phantomsection\label{refs}
\begin{CSLReferences}{1}{0}
\bibitem[\citeproctext]{ref-ambwaniInterpersonalDysfunctionAffectregulation2014}
Ambwani, S., Slane, J. D., Thomas, K. M., Hopwood, C. J., \& Grilo, C.
M. (2014). Interpersonal dysfunction and affect-regulation difficulties
in disordered eating among men and women. \emph{Eating Behaviors},
\emph{15}(4), 550--554.
\url{https://doi.org/10.1016/j.eatbeh.2014.08.005}

\bibitem[\citeproctext]{ref-farahaniApplicationGraphTheory2019}
Farahani, F., Karwowski, W., \& Lighthall, N. (2019). Application of
{Graph Theory} for {Identifying Connectivity Patterns} in {Human Brain
Networks}: {A Systematic Review}. \emph{Frontiers in Neuroscience},
\emph{13}, 585. \url{https://doi.org/10.3389/fnins.2019.00585}

\bibitem[\citeproctext]{ref-isvoranuNetworkPsychometricsGuide2022}
Isvoranu, A.-M., Epskamp, S., Waldorp, L., \& Borsboom, D. (Eds.).
(2022). \emph{Network {Psychometrics} with {R}: {A Guide} for
{Behavioral} and {Social Scientists}}. Routledge.
\url{https://doi.org/10.4324/9781003111238}

\bibitem[\citeproctext]{ref-kan2019}
Kan, K.-J., Maas, H. L. J. van der, \& Levine, S. Z. (2019). Extending
psychometric network analysis: Empirical evidence against {\emph{g}} in
favor of mutualism? \emph{Intelligence}, \emph{73}, 52--62.
\url{https://doi.org/10.1016/j.intell.2018.12.004}

\bibitem[\citeproctext]{ref-lauritzenGraphicalModels1996a}
Lauritzen, S. L. (1996). \emph{Graphical {Models}}. Clarendon Press.
\url{https://books.google.com?id=mGQWkx4guhAC}

\bibitem[\citeproctext]{ref-lewisTastePrivacyAnalysis2008}
Lewis, K., Kaufman, J., \& Christakis, N. (2008). The {Taste} for
{Privacy}: {An Analysis} of {College Student Privacy Settings} in an
{Online Social Network}. \emph{Journal of Computer-Mediated
Communication}, \emph{14}(1), 79--100.
\url{https://doi.org/10.1111/j.1083-6101.2008.01432.x}

\bibitem[\citeproctext]{ref-murphyInterpersonalPsychotherapyEating2012}
Murphy, R., Straebler, S., Basden, S., Cooper, Z., \& Fairburn, C. G.
(2012). Interpersonal {Psychotherapy} for {Eating Disorders}.
\emph{Clinical Psychology \& Psychotherapy}, \emph{19}(2), 150--158.
\url{https://doi.org/10.1002/cpp.1780}

\bibitem[\citeproctext]{ref-zhangLongitudinalNetworkAnalysis2024a}
Zhang, J., Cui, S., Zickgraf, H. F., Barnhart, W. R., Xu, Y., Wang, Z.,
Ji, F., Chen, G., \& He, J. (2024). A {Longitudinal Network Analysis} of
{Emotion Regulation}, {Interpersonal Problems}, and {Eating Disorder
Psychopathology} in {Chinese Adolescents}. \emph{International Journal
of Eating Disorders}, \emph{57}(12), 2415--2426.
\url{https://doi.org/10.1002/eat.24292}

\end{CSLReferences}




\end{document}
